\documentclass{beamer}

\usepackage[utf8]{inputenc}
\usepackage[spanish]{babel}
\usepackage{amsmath}
\usepackage[nosetup]{evan}
%\usetheme{Goddard}
\usetheme{Madrid}
\hypersetup{colorlinks,allcolors=.,urlcolor=magenta}
\usepackage[table]{xcolor} % Para definir colores en tablas
\usepackage{graphicx} % Para redimensionar la tabla

\title{Investigación de Operaciones II}
\subtitle{Unidad 2: Introducción a las Cadenas de Markov\\Cadenas de Markov absorbentes y tiempo de primera pasada}
\author[Ricardo Largaespada]{Ricardo Jesús Largaespada Fernández}
\institute[UNI]{Ingeniería de Sistemas, DACTIC, UNI}
\date{24 de Abril, 2025}

\begin{document}
\begin{frame}{Inicio (15 min)}  
\begin{itemize}
    \item Pregunta detonante + animación de un proceso donde el cliente navega entre servicios  
          \(\Rightarrow\) llega a un \alert{estado final} (abandono o éxito).
    \item Cuestiones clave:  
          \(\bigl\lbrace\)¿Cuánto tarda en llegar? \quad ¿Puede salir del estado final?\(\bigr\rbrace\)
    \item Se introducen:  
        \begin{itemize}
            \item \textit{Estado absorbente}: \(p_{ii}=1\).
            \item \textit{Tiempo de primera pasada} \(T_{ij}=\min\{n\ge1 : X_n=j\,|\,X_0=i\}\).
        \end{itemize}
\end{itemize}

\end{frame}

\begin{frame}{Desarrollo (65 min): “estaciones” en rotación}
\begin{enumerate}
    \item \textbf{Estación 1:} localizar estados absorbentes y reordenar la matriz de transición  
          \[
              P=\begin{pmatrix}
                     Q & R\\[2pt]
                     0 & I
                 \end{pmatrix}\!.
          \]
    \item \textbf{Estación 2:} cálculo del \emph{tiempo esperado de primera pasada}  
          \[
              N=(I-Q)^{-1}, \qquad 
              \bm{t}=N\bm{1}, \qquad 
              t_i=\sum_{j} n_{ij}.
          \]
    \item \textbf{Estación 3:} simulaciones (abandono de cliente, fallo de sistemas…) y toma de decisiones a partir de \(\bm{t}\).
\end{enumerate}
\end{frame}

\begin{frame}{Cierre (20 min)}  
\begin{itemize}
    \item Construcción colectiva de tabla‐resumen:  
        \(\{\)identificación de absorbentes,\; cálculo y significado de \(T_{ij}\),\; aplicaciones en Ingeniería de Sistemas\(\}\).
    \item Pregunta abierta (reflexión individual):\\
          “\emph{¿En qué procesos digitales sería útil conocer el tiempo de absorción o permanencia?}”
\end{itemize}

\vfill
\centering
\scriptsize
\(\displaystyle
\text{Matriz clave: } \Pr\!\bigl[T_{ij}<\infty\bigr]=\begin{cases}
1,&\text{\(j\) absorbente y la cadena es regular},\\
<1,&\text{caso contrario}.
\end{cases}
\)
\end{frame}

\end{document}

