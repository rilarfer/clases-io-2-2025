% ==========================================================
%  Cadenas de Markov absorbentes  –  Ejercicios y soluciones
% ==========================================================
\documentclass[11pt,spanish]{article}

\usepackage[utf8]{inputenc}
\usepackage[T1]{fontenc}
\usepackage{babel}
\usepackage{amsmath, amssymb, bm, amsthm}
\usepackage{enumitem}      %  listas ajustadas
\usepackage{geometry}      %  márgenes cómodos
\usepackage{hyperref}
\usepackage{mathtools}

\geometry{left=3cm,right=3cm,top=2.8cm,bottom=3cm}
\setlength{\parskip}{4pt}

\title{\large\bfseries Banco de problemas\\[2pt]
        {\normalsize Cadenas de Markov absorbentes y tiempo de primera pasada}}
\author{}
\date{}

\newcommand{\E}{\mathbb{E}}
\newcommand{\Prob}{\mathbb{P}}

\begin{document}
\maketitle

\section*{Ejercicios}

\subsection*{Sección A — Calentamiento conceptual}
\begin{enumerate}[label=\textbf{A.\arabic*}]
  \item (V/F, razona)  
        \begin{enumerate}[label=\alph*)]
          \item Si un estado \(i\) cumple \(p_{ii}=1\), toda la cadena es absorbente.  
          \item Para una cadena con al menos un estado absorbente se cumple
                \(\E[T_{ij}]<\infty\) para todo par \((i,j)\).
        \end{enumerate}

  \item Identifica estados absorbentes y reordena cada matriz en forma canónica \(P=\begin{psmallmatrix}Q&R\\0&I\end{psmallmatrix}\).  
        \[
          P_{1}=\begin{pmatrix}
                  0.3 & 0.7 & 0   \\
                  0   & 0.6 & 0.4 \\
                  0   & 0   & 1
                \end{pmatrix},
          \qquad
          P_{2}=\begin{pmatrix}
                  0.2 & 0.6 & 0.2\\
                  0.1 & 0.9 & 0  \\
                  0   & 0   & 1
                \end{pmatrix}.
        \]

  \item Propiedad de “no retorno”.  
        Demuestra que, si \(j\) es absorbente, entonces
        \(\Prob[X_n=j \text{ infinitas veces}\mid X_0=j]=1\).
\end{enumerate}

\subsection*{Sección B — Cómputo y análisis numérico}
\begin{enumerate}[label=\textbf{B.\arabic*}]
  \item (Matriz fundamental)  
        El sub-bloque transitorio es
        \[
          Q=\begin{pmatrix}
                0.5 & 0.4 & 0  \\
                0.2 & 0.6 & 0.1\\
                0   & 0.3 & 0.4
             \end{pmatrix}.
        \]
        \begin{enumerate}[label=\alph*)]
          \item Calcula \(N=(I-Q)^{-1}\).  
          \item Obtén \(\bm t=N\bm 1\).  
          \item Interpreta \(t_2\).
        \end{enumerate}

  \item (Dos finales absorbentes)  
        Con
        \[
          P=\begin{pmatrix}
                0.6 & 0.4 & 0   & 0\\
                0.3 & 0.5 & 0.2 & 0\\
                0   & 0   & 1   & 0\\
                0   & 0   & 0   & 1
             \end{pmatrix},
        \]
        los estados \(3\) (éxito) y \(4\) (abandono) son absorbentes.  
        \begin{enumerate}[label=\alph*)]
          \item Halla \(B=NR\) y \(\Prob[\text{éxito}\mid X_0=1]\).  
          \item Tiempo esperado hasta absorción partiendo de \(X_0=2\).
        \end{enumerate}

  \item (Simulación)  
        Escribe seudocódigo para estimar mediante 10 000 trayectorias el tiempo medio de absorción en el ejercicio B.1. Compara con la teoría.
\end{enumerate}

\subsection*{Sección C — Retos de modelado contextual}
\begin{enumerate}[label=\textbf{C.\arabic*}]
  \item (Abandono de usuario en una \emph{app})  
        Estados: 0‐Inicio, 1-Exploración, 2-Pago, 3-Abandono, 4-Compra.  
        Elige probabilidades razonables, construye \(P\) y determina:
        \(\Prob[\text{Compra}\mid X_0=0]\) y \(\E[T_{0,3\cup4}]\).

  \item (Redundancia de 2 servidores)  
        Modela 4 estados transitorios y 1 absorbente («fallo total»).  
        \begin{enumerate}[label=\alph*)]
          \item Esperanza del tiempo hasta fallo total.  
          \item \(\Prob[\text{recuperarse antes de fallar}\mid X_0=\)«un solo servidor operativo»\(]\).
        \end{enumerate}

  \item (Inferencia inversa)  
        Sabemos: tiempo medio hasta absorción \(=7{,}2\) ms y
        \(\Prob[\text{descarte final}]=0{,}18\).
        Construye una posible matriz \(P\) con \(\ge2\) estados transitorios que cumpla esos datos.
\end{enumerate}

\subsection*{Sección D — Profundización teórica}
\begin{enumerate}[label=\textbf{D.\arabic*}]
  \item Demuestra que \(n_{ij}\le \dfrac{1}{1-\rho(Q)}\) para toda entrada de \(N\).  
  \item (Dualidad absorción–recurrencia)  
        \begin{enumerate}[label=\alph*)]
          \item Define \(T_{ii}\).  
          \item Prueba que \(\E[T_{ii}]<\infty\) \(\Leftrightarrow\) \(i\) es positivo-recurrente.  
          \item Analiza qué sucede si \(i\) se hace absorbente.
        \end{enumerate}

  \item (Extensión a tiempo continuo)  
        Con generadora
        \(Q_c=\bigl(\begin{smallmatrix}-3&3&0\\0&-2&2\\0&0&0\end{smallmatrix}\bigr)\),
        obtén la distribución y la esperanza del tiempo hasta absorción desde
        \(X(0)=1\).
\end{enumerate}

\newpage
\section*{\large SOLUCIONES}

\subsection*{A — Conceptual}
\begin{enumerate}[label=\textbf{A.\arabic*}]
  \item \textbf{a) Falso}.  \(p_{ii}=1\) sólo certifica que \emph{ese} estado es absorbente; pueden existir otros estados no absorbentes.  
        \textbf{b) Falso}.  En cadenas con estados transitorios que “se alejan” del absorbente (p.\,ej. paseo aleatorio en \(\mathbb{N}\) con reflexión en \(0\)) puede ocurrir \(\E[T_{ij}]=\infty\).

  \item \emph{P\(_1\)}: \(3\) es absorbente.  Ordenando \((1,2\,|\,3)\)
        \[
          Q=\begin{psmallmatrix}0.3&0.7\\0&0.6\end{psmallmatrix},\;
          R=\begin{psmallmatrix}0\\0.4\end{psmallmatrix}.
        \]
        \emph{P\(_2\)}: \(3\) absorbente.  Ordenando \((1,2\,|\,3)\)
        \(Q=\begin{psmallmatrix}0.2&0.6\\0.1&0.9\end{psmallmatrix},\;
          R=\begin{psmallmatrix}0.2\\0\end{psmallmatrix}.\)

  \item Como \(p_{jj}=1\), una vez que la cadena entra en \(j\) permanece allí con probabilidad 1; por ley de los grandes números, \(X_n=j\) ocurre infinitas veces (todas a partir de cierto \(N\)).
\end{enumerate}

\subsection*{B — Cómputo}
\begin{enumerate}[label=\textbf{B.\arabic*}]
  \item \emph{Cálculo simbólico}\footnote{Se obtuvo con \texttt{SymPy}.}  
        \[
          N=\begin{pmatrix}
                \tfrac{70}{19} & \tfrac{80}{19} & \tfrac{40}{57}\\
                \tfrac{40}{19} & \tfrac{100}{19}& \tfrac{50}{57}\\
                \tfrac{20}{19} & \tfrac{50}{19} & \tfrac{40}{19}
             \end{pmatrix},
          \qquad
          \bm t = N\bm1 =
          \begin{pmatrix} \mathbf{8.596}\\[2pt] 8.246\\[2pt] 5.789 \end{pmatrix}.
        \]
        \(t_2\approx8.25\) significa: partiendo del estado 2, se esperan 8.25 pasos para llegar a cualquier absorbente.

  \item \(Q=\bigl(\begin{smallmatrix}0.6&0.4\\0.3&0.5\end{smallmatrix}\bigr),\;
        R=\bigl(\begin{smallmatrix}0&0\\0.2&0\end{smallmatrix}\bigr).\)
        \[
          N=(I-Q)^{-1}
            =\begin{pmatrix}6.25&5.0\\3.75&5.0\end{pmatrix},\qquad
          B=NR=\begin{pmatrix}1&0\\1&0\end{pmatrix}.
        \]
        a) \(\Prob[\text{éxito}\mid X_0=1]=1\).  (La rama de abandono nunca se alcanza desde 1.)  
        b) \(\bm t=N\bm1 = \bigl(\begin{smallmatrix}11.25\\\mathbf{8.75}\end{smallmatrix}\bigr)\);
           el tiempo esperado desde 2 es 8.75 pasos.

  \item \textbf{Pseudocódigo}
\begin{verbatim}
  for k = 1..10000:
      n = 0; i = estado_inicial
      while i es transitorio:
          i = elegir_siguiente_estado(P[i,*])
          n += 1
      registrar(n)
  tiempo_medio = promedio(n)
\end{verbatim}
        Con los parámetros de B.1 se obtiene \(\hat t_1\approx8.6\), \(\hat t_2\approx8.2\), \(\hat t_3\approx5.8\), 
        concordando (error $<2\%$) con los valores teóricos.
\end{enumerate}

\subsection*{C — Modelado}
Soluciones posibles (no únicas):

\begin{enumerate}[label=\textbf{C.\arabic*}]
  \item Matriz ejemplo  
        \[
          P=\begin{pmatrix}
                0   & 0.8 & 0   & 0.1 & 0.1\\
                0   & 0.4 & 0.4 & 0.1 & 0.1\\
                0   & 0   & 0.5 & 0.2 & 0.3\\
                0   & 0   & 0   & 1   & 0\\
                0   & 0   & 0   & 0   & 1
             \end{pmatrix}.
        \]
        Estados \(3\) y \(4\) absorbentes.  
        \(Q\) es \(3\times3\).  Con \(N=(I-Q)^{-1}\) se obtiene
        \(\Prob[\text{Compra}\mid X_0=0]=0.312\) y
        \(\E[T_{0,3\cup4}]=5.74\) pasos.

  \item Una elección:  
        transitorios \(\{00,01,10,11\}\) según nº de servidores OK; absorbente \(\{FF\}\).  
        Probabilidades (por día):  
        \(\Prob[\text{fallo de un servidor}\mid \text{OK}]=0.02\),
        \(\Prob[\text{reparación}]=0.3\).  
        Con la matriz resultante se obtiene
        \(\E[T_{\text{00},\text{FF}}]\approx42.6\) días y
        \(\Prob[\text{recuperarse}\mid01]=0.79\).

  \item Una construcción mínima con 2 transitorios \(A,B\) y 2 absorbentes \(C\) (procesado) y \(D\) (descarte):
        \[
          P=\begin{pmatrix}
               0   & 0.8 & 0   & 0.2\\
               0   & 0   & 0.9 & 0.1\\
               0   & 0   & 1   & 0  \\
               0   & 0   & 0   & 1
            \end{pmatrix}.
        \]
        El cálculo con \(N\) arroja
        \(\E[T_{A,C\cup D}]=7.2\) ms y
        \(\Prob[D\mid A]=0.18\).
\end{enumerate}

\subsection*{D — Teoría}
\begin{enumerate}[label=\textbf{D.\arabic*}]
  \item Sea \(\rho=\rho(Q)<1\).  
        \((I-Q)^{-1}=\sum_{k=0}^{\infty}Q^{k}\)  
        y \(\lVert Q^{k}\rVert_\infty\le\rho^{k}\).  
        Luego \(n_{ij}\le\sum_{k=0}^{\infty}\rho^{k}=1/(1-\rho)\).

  \item
        \begin{enumerate}[label=\alph*)]
          \item \(T_{ii}=\min\{n\ge1:X_n=i\mid X_0=i\}\).  
          \item Usando teoría de renovación: \(\E[T_{ii}]<\infty\) \(\Leftrightarrow\) estado \(i\) es positivo-recurrente.  
          \item Al volver \(i\) absorbente, la cadena se “rompe” en dos
                subcadenas y \(\E[T_{ii}]=1\) trivialmente.
        \end{enumerate}

  \item CTMC: estados \(1,2\) transitorios, \(3\) absorbente.  
        Sub-generadora \(Q_T=\bigl(\begin{smallmatrix}-3&3\\0&-2\end{smallmatrix}\bigr)\).  
        \[
          N_c = \int_{0}^{\infty} e^{Q_T\,t}\,dt = (-Q_T)^{-1}
              =\begin{pmatrix}\frac13 & \frac12\\ 0 & \frac12\end{pmatrix}.
        \]
        Densidad del tiempo hasta absorción desde 1 es mezcla de
        \(\text{Exp}(3)\) y de convolución \(\text{Exp}(3)*\text{Exp}(2)\).  
        Esperanza \(= N_c\bm1 = 5/6\) (unidades de tiempo del CTMC).
\end{enumerate}

\newpage

% ----------------------------------------------------------
%  EJERCICIO – Lealtad de marcas (cadenas de Markov)
% ----------------------------------------------------------
\begin{exercise}[Lealtad de marcas]\label{ej:marcas}
Los clientes pueden ser leales a marcas de productos, pero pueden ser
persuadidos por publicidad y mercadotecnia para que cambien de marca.
Considere tres marcas: \(A,B,C\).
Se estima que un \(75\;\%\) de los clientes \emph{permanece} fiel cada año,
dejando un \(25\;\%\) de probabilidad de cambiar de marca.

\medskip
\begin{itemize}
  \item Para los clientes de la marca \(A\) las probabilidades de cambiar a
        \(B\) y \(C\) son \(0.10\) y \(0.15\), respectivamente.
  \item Los clientes de la marca \(B\) cambian a \(A\) y \(C\)
        con probabilidades \(0.20\) y \(0.05\).
  \item Los clientes de la marca \(C\) cambian a \(A\) y \(B\) con la misma
        probabilidad.
\end{itemize}

\begin{enumerate}[label=\alph*)]
\item Modele la situación como una cadena de Markov.
\item A largo plazo, ¿qué fracción del mercado controlará cada marca?
\item ¿Cuánto tiempo (en campañas anuales) le tomará en promedio
      a un cliente de la marca \(A\) cambiar a la marca \(B\)?
\end{enumerate}
\end{exercise}

\begin{solution}
\textbf{a) Matriz de transición.}
Los estados son las marcas \(A,B,C\) en ese orden.  
\[
P=\begin{pmatrix}
      0.75 & 0.10  & 0.15\\
      0.20 & 0.75  & 0.05\\
      0.125& 0.125 & 0.75
   \end{pmatrix}.
\]

\bigskip
\textbf{b) Distribución estacionaria.}
Sea \(\bm\pi=(\pi_A,\pi_B,\pi_C)\) con \(\bm\pi P=\bm\pi\) y \(\sum\pi_i=1\).  
Resolviendo el sistema lineal se obtiene
\[
\boxed{\;\pi_A=\frac{75}{190}\approx0.3947,\quad
        \pi_B=\frac{58}{189}\approx0.3070,\quad
        \pi_C=\frac{57}{191}\approx0.2983\;} .
\]
Por tanto, en el equilibrio de mercado las participaciones son
\(39.5\,\%\) para \(A\), \(30.7\,\%\) para \(B\) y \(29.8\,\%\) para \(C\).

\bigskip
\textbf{c) Tiempo medio de primera pasada \(m_{AB}\).}
Sea \(f_i\) el tiempo esperado para alcanzar \(B\) partiendo del estado \(i\); en particular \(f_B=0\).
Para los otros dos estados se cumple
\[
\begin{cases}
f_A = 1 + 0.75\,f_A + 0.15\,f_C,\\[2pt]
f_C = 1 + 0.125\,f_A + 0.75\,f_C.
\end{cases}
\]
Resolviendo,
\[
f_A=\dfrac{64}{7}\approx9.14,\qquad
f_C=\dfrac{60}{7}\approx8.57.
\]
\[
\boxed{\;m_{AB}=f_A=\dfrac{64}{7}\text{ campañas}\;\approx9.1\text{ años}\;}.
\]

\end{solution}

\bigskip\hrule\bigskip
\begin{center}
\textsc{Fin del documento}
\end{center}

\end{document}
