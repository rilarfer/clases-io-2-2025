\documentclass{article}

% Matemática y símbolos
\usepackage{amsmath}
\usepackage{amssymb}

% Márgenes y geometría de la página
\usepackage{geometry}
\geometry{a4paper, margin=1in}

% Colores y cajas
\usepackage{xcolor}
\usepackage{mdframed}

% Gráficos y diagramas
\usepackage{tikz}
\usetikzlibrary{positioning}  % Para posicionamiento relativo en TikZ

% Tablas elegantes y columnas personalizadas
\usepackage{booktabs}
\usepackage{array}
\usepackage{float}

% Imágenes
\usepackage{graphicx}

% Estilo de párrafo
\usepackage{parskip}

% Idioma
\usepackage[spanish]{babel}
\usepackage[utf8]{inputenc}  % Si no usas UTF-8 nativo

% Captions
\usepackage{caption}

% Información del documento
\title{Trabajo 10: Toma de Decisiones}
\author{Ricardo Largaespada}
\date{10 de junio 2025}

% ---------- Entorno "problema" ----------
\newenvironment{problem}[1]{%
  \section*{Problema #1}\addcontentsline{toc}{section}{Problema #1}%
  \hrule\vspace{0.75em}%
}{\vspace{0.75em}\hrule}

\begin{document}

\maketitle
\bigskip

%----------------------------------------------------------------------
\begin{problem}{15.2-4}
\textbf{Tema: Criterios sin probabilidades (Maximin, Maximax, Laplace, Regret).}

\begin{center}
% TODO: Sustituir esta matriz por la del libro
\begin{tabular}{lccc}
\toprule
 & $s_1$ & $s_2$ & $s_3$\\ \midrule
$A_1$ &  &  &  \\
$A_2$ &  &  &  \\ 
$A_3$ &  &  &  \\ \bottomrule
\end{tabular}
\end{center}

\begin{enumerate}
  \item Aplicar el criterio Maximax.  
  \item Aplicar el criterio Maximin.  
  \item Aplicar el criterio de Laplace y el de Arrepentimiento (Minimax).  
\end{enumerate}
\end{problem}

%----------------------------------------------------------------------
\begin{problem}{15.2-3}
\textbf{Tema: Criterio de Bayes (Valor Esperado Monetario).}

% TODO: Pegar aquí tabla de pagos y las probabilidades a priori
\begin{center}\fbox{Tabla de pagos con $p_j$}\end{center}

\begin{enumerate}
  \item[\textbf{(b)}] Calcular el VEM de cada alternativa.  
  \item[\textbf{(c)}] Elegir la mejor alternativa según VEM.  
  \item[\textbf{(d)}] Analizar cómo cambia la decisión si $p_1$ aumenta a \dots  
\end{enumerate}
\end{problem}

%----------------------------------------------------------------------
\begin{problem}{15.3-1}
\textbf{Tema: Árbol de decisión simple.}

% TODO: Insertar diagrama del libro o describir las ramas (pagos, probabilidades)
\vspace{1em}
\noindent\textit{Instrucciones:} Dibuje el árbol, realice \emph{rollback} y determine la política óptima.
\end{problem}

%----------------------------------------------------------------------
\begin{problem}{15.3-3}
\textbf{Tema: Valor de la información (VEPI/VEI).}

% TODO: Árbol “sin estudio” y árbol “con estudio” con su costo
\vspace{1em}
\noindent\textit{Solicitado:}  
\begin{enumerate}
  \item Resolver el árbol sin información adicional.  
  \item Resolver el árbol con información (estudio).  
  \item Calcular $\text{VEI}$ y comparar con el costo del estudio.  
\end{enumerate}
\end{problem}

%----------------------------------------------------------------------
\begin{problem}{15.2-5}
\textbf{Tema: Análisis de sensibilidad (probabilidad de ruptura).}

% TODO: Tabla de pagos + probabilidad a variar
\begin{enumerate}
  \item Encontrar el valor crítico de $p$ para el cual las alternativas $A_1$ y $A_2$ son indiferentes.  
  \item Graficar VEM vs.\ $p$ en el intervalo $[0,1]$.  
\end{enumerate}
\end{problem}

%----------------------------------------------------------------------
\begin{problem}{15.6-5}
\textbf{Tema: Teoría de la utilidad (función exponencial).}

% TODO: Describir lotería, pagos y parámetro $R$
\begin{enumerate}
  \item Calcular la utilidad esperada (VEU) de cada alternativa.  
  \item Determinar el equivalente de certeza.  
  \item Comparar con la decisión basada en VEM.  
\end{enumerate}
\end{problem}

%----------------------------------------------------------------------
\begin{problem}{15.5-5}
\textbf{Tema: Modelo de un período (demanda y costos).}

% TODO: Indicar $\theta$, $C_u$, $C_o$ o tabla discreta de demanda
\begin{enumerate}
  \item Calcular la cantidad óptima $Q^{*}$.  
  \item Estimar la utilidad/costo esperado y el nivel de servicio.  
\end{enumerate}
\end{problem}

%----------------------------------------------------------------------
\begin{problem}{15.4-5}
\textbf{Tema: Árbol multiperíodo / inversión secuencial.}

% TODO: Proporcionar diagrama o tabla de flujos de caja y probabilidades por año
\vspace{1em}
Resolver el árbol completo con \emph{rollback}; indicar la estrategia óptima de reinversión o venta.
\end{problem}

%----------------------------------------------------------------------
\begin{problem}{15.1-2}
\textbf{Tema: Factor Rating / AHP resumido.}

% TODO: Incluir pesos de criterios y calificaciones de alternativas
\begin{enumerate}
  \item Construir la matriz de ponderaciones y calcular la puntuación compuesta de cada alternativa.  
  \item Indicar la alternativa preferida.  
\end{enumerate}
\end{problem}

%======================================================================
\end{document}