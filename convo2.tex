\documentclass[11pt,paper=a4,answers, addpoints]{exam}
\usepackage{graphicx,lastpage,comment}
\usepackage{upgreek}
\usepackage{censor}
\censorruledepth=-.2ex
\censorruleheight=.1ex
\hyphenpenalty 10000
\usepackage[paperheight=10.5in,paperwidth=8.27in,bindingoffset=0in,left=0.8in,right=1in, top=1.3in,bottom=1in,headsep=.5\baselineskip]{geometry}
\flushbottom
\usepackage[normalem]{ulem}
\usepackage[utf8]{inputenc}
\usepackage[spanish]{babel}
\renewcommand\ULthickness{2pt} %%---> For changing thickness of underline
\setlength\ULdepth{1.5ex}%\maxdimen ---> For changing depth of underline
\renewcommand{\baselinestretch}{1}
\pagestyle{empty}

\pagestyle{headandfoot}
\headrule
\newcommand{\continuedmessage}{%
  \ifcontinuation{\footnotesize Pregunta \ContinuedQuestion\ continua\ldots}{}%
}
\runningheader{\footnotesize Segunda Convocatoria}
{\footnotesize Investigación de Operaciones I}
{\footnotesize Página \thepage\ de \numpages}
\footrule
\footer{\footnotesize }
{}
{\ifincomplete{\footnotesize Pregunta \IncompleteQuestion\ continua en la siguiente página \ldots}
  {\iflastpage{\footnotesize Final del examen}{\footnotesize Por favor vea la siguiente página\ldots}}}
\usepackage{amsfonts,amsmath}
\usepackage{cleveref}
\crefname{figure}{figure}{figures}
\crefname{question}{question}{questions}
\renewcommand\thequestion{\arabic{question}}
\renewcommand{\questionlabel}{\thequestion)}
\renewcommand{\questionshook}{%
  \setlength{\leftmargin}{0pt}%
  \setlength{\labelwidth}{-\labelsep}%
}
\usepackage{amsmath}
\decimalpoint
\nopointsinmargin
\pointpoints{Punto}{Punto}

\marginpointname{\points}
\pointformat{\boldmath\themarginpoints}
\bracketedpoints
\usepackage{quoting,xparse}

\pointpoints{punto}{puntos}
\bonuspointpoints{punto extra}{puntos extra}
 
\totalformat{Pregunta \thequestion: \totalpoints puntos}
 
\hqword{Pregunta}
\hpgword{Página}
\hpword{Puntos}
\hsword{Puntos obtenidos}
\htword{Total}
\usepackage{circuitikz}
\usepackage{color}
\usepackage{pgfplots,graphicx}

\usepackage{ mathrsfs}
\newcommand{\Laplace}[1]{\ensuremath{\mathscr{L}{\left\lbrace #1\right\rbrace}}}
\newcommand{\InvLap}[1]{\ensuremath{\mathscr{L}^{-1}{\left\lbrace #1\right\rbrace}}}

\usepackage{background}
\usepackage{lipsum} % Para texto de ejemplo, puedes eliminarlo

% Configuración del fondo
\backgroundsetup{
  scale=1,  % Escala del PDF
  color=black,  % Color (puedes usar transparente si es un PDF a color)
  opacity=1,  % Opacidad (1 es opaco, 0 es transparente)
  angle=0,  % Ángulo del PDF
  pages=all,  % Aplicar a todas las páginas
  position=current page.south west,  % Posición del PDF
  nodeanchor=south west,
  vshift=0mm,  % Desplazamiento vertical
  hshift=0mm,  % Desplazamiento horizontal
  contents={\includegraphics[width=\paperwidth,height=\paperheight]{fondo.pdf}} % PDF de fondo
}

\begin{document}
%\printanswers
\noprintanswers
%\shadedsolutions
%\fillwithdottedlines
\shorthandoff{<>}
\thispagestyle{empty}
\begin{center}
    \textit{\textbf{Examen de Segunda Convocatoria}}
\end{center}
\noindent
\vspace{-.1cm}
\begin{minipage}[t]{.6\textwidth}%
  {\bfseries Nombres}: \makebox[.75\textwidth]{\hrulefill} \par
  {\bfseries Apellidos}: \makebox[.75\textwidth]{\hrulefill} \par
  {\bfseries Docente}: Ricardo Largaespada
\end{minipage}%
\hfill
\begin{minipage}[t]{.4\textwidth}%
  {\bfseries Curso}: Investigación de Operaciones I \par
  {\bfseries Fecha}: 13 de diciembre 2024 \par
  {\bfseries Grupo:} 
  \vspace{1ex}
\end{minipage}    \hfill
  \rule{\textwidth}{1pt}
\begin{comment}
\noindent\underline{\sc Instrucciones a estudiantes}
\vspace{5mm}
\begin{enumerate}
\item Este examen contiene {\bf CINCO (5)} preguntas.
\item Puedes utilizar calculadora. Sin embargo, debes de escribir sistemáticamente los pasos del trabajo.
\item Entrega todo el trabajo adicional en páginas rotuladas.
\end{enumerate}

\begin{center}
\textbf{Tabla (para uso EXCLUSIVO del profesor)}\\
\addpoints
\combinedgradetable[v][questions]
\end{center}
	
\noindent
\rule[1ex]{\textwidth}{1pt}
\end{comment}

\begin{questions}
\titledquestion{Método Gráfico}[20]

Una empresa produce dos productos, \(P_1\) y \(P_2\). Cada producto requiere recursos limitados de los departamentos de Producción y Ensamblaje. Las ganancias por unidad de \(P_1\) y \(P_2\) son 50 y 40 unidades monetarias, respectivamente. Las restricciones de recursos son:

\[
3P_1 + 2P_2 \leq 60 \quad \text{(Producción)},
\]
\[
2P_1 + 4P_2 \leq 80 \quad \text{(Ensamblaje)},
\]
\[
P_1, P_2 \geq 0.
\]

La función objetivo es:
\[
\text{Maximizar } Z = 50P_1 + 40P_2.
\]

Resuelve:
\begin{enumerate}
    \item Representa gráficamente las restricciones.
    \item Determina la región factible.
    \item Encuentra la solución óptima.
\end{enumerate}

\titledquestion{Programación Lineal - Método Tabular}[20]

Una empresa fabrica tres tipos de muebles: sillas (\(S\)), mesas (\(M\)) y escritorios (\(E\)). La función objetivo es maximizar las ganancias, con las siguientes ganancias unitarias:
\[
\text{Ganancias: } Z = 30S + 40M + 35E.
\]

Las restricciones son las siguientes:
\[
S + M + E \leq 100 \quad \text{(Disponibilidad de madera en unidades)},
\]
\[
2S + M + 3E \leq 180 \quad \text{(Horas de trabajo disponibles)},
\]
\[
S, M, E \geq 0.
\]

Resuelve utilizando el {\bf método tabular}:
\begin{enumerate}
    \item Escribe el modelo en forma estándar.
    \item Encuentra la solución óptima mostrando cada iteración.
\end{enumerate}

\titledquestion{Dualidad - Interpretación Económica}[20]

Un agricultor tiene 10 hectáreas de tierra y desea maximizar sus ingresos plantando maíz y frijol. El beneficio por hectárea es de 2000 unidades monetarias para el maíz y 3000 unidades monetarias para el frijol. Las restricciones son las siguientes:
\[
2M + F \leq 12 \quad \text{(Disponibilidad de agua en miles de litros)},
\]
\[
M + F \leq 10 \quad \text{(Área disponible en hectáreas)},
\]
\[
M, F \geq 0.
\]

La función objetivo es:
\[
\text{Maximizar } Z = 2000M + 3000F.
\]

Resuelve:
\begin{enumerate}
    \item Plantea el problema dual asociado.
    \item Encuentra la solución del problema dual.
    \item Interpreta económicamente los valores de las variables duales (precios sombra).
\end{enumerate}

\titledquestion{Transporte (Balanceado}[20]

Una empresa necesita transportar productos desde tres plantas \(A, B, C\) hasta tres almacenes \(X, Y, Z\). Las capacidades de oferta y demanda están balanceadas. Los costos unitarios de transporte (en unidades monetarias) son los siguientes:

\[
\begin{array}{c|c c c c}
 & X & Y & Z & \text{Oferta} \\
\hline
A & 4 & 8 & 8 & 20 \\
B & 6 & 4 & 5 & 30 \\
C & 7 & 6 & 3 & 25 \\
\hline
\text{Demanda} & 15 & 35 & 25 &
\end{array}
\]

\begin{parts}
    \part Encuentra una solución inicial usando el método de la Esquina Noroeste.
    \part Encuentra la solución óptima usando el método de transporte.
\end{parts}

\titledquestion{Asignación}[20]

Una empresa asigna cuatro empleados \(A, B, C, D\) a cuatro tareas \(1, 2, 3, 4\). Los costos asociados (en unidades monetarias) de asignar cada empleado a una tarea están dados por la siguiente matriz:

\[
\begin{array}{c|c c c c}
 & \text{Tarea 1} & \text{Tarea 2} & \text{Tarea 3} & \text{Tarea 4} \\
\hline
A & 8 & 6 & 7 & 11 \\
B & 8 & 3 & 4 & 6 \\
C & 8 & 9 & 8 & 5 \\
D & 9 & 4 & 6 & 4
\end{array}
\]

\begin{parts}
    \part Usa el método de asignación (algoritmo de Húngaro) y encuentra la asignación óptima.
    \part Indica el costo mínimo total.
\end{parts}

\end{questions}

\end{document}