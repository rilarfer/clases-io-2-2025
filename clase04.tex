\documentclass{beamer}

\usepackage[utf8]{inputenc}
\usepackage[spanish]{babel}
\usepackage{amsmath}
\usepackage{amssymb}
\usepackage{tikz}
\usetheme{Goddard}
\hypersetup{colorlinks,allcolors=.,urlcolor=magenta}

\title{Investigación de Operaciones II}
\subtitle{Unidad 1: Análisis de Redes}
\author{Ricardo Jesús Largaespada Fernández}
\institute{Ingeniería de Sistemas, DACTIC, UNI}
\date{11 de Marzo, 2025}

\begin{document}

\frame{\titlepage}

\begin{frame}{Agenda}
    \tableofcontents
\end{frame}

\section{Sesión 4}
\begin{frame}{Sesión 4}

\textbf{Tema}
\begin{enumerate}
    \item Árbol de Expansión Mínima (MST)
\end{enumerate}

\textbf{Objetivo}
\begin{itemize}
    \item El estudiante aplicará los algoritmos de Kruskal y Prim para determinar el Árbol de Expansión Mínima de un grafo ponderado, optimizando la conexión en redes con costo mínimo.
\end{itemize}

\end{frame}

\subsection{Conceptos Básicos}

\begin{frame}{Árbol de Expansión Mínima (MST)}

\textbf{Definición:}  
Es un subconjunto de las aristas de un grafo conexo, ponderado y no dirigido, que conecta todos los vértices sin ciclos y con el mínimo peso total posible.

\textbf{Características:}
\begin{itemize}
    \item El MST tiene exactamente $n-1$ aristas (donde $n$ es el número de vértices).
    \item No existen ciclos en el MST.
    \item Puede no ser único si varias aristas tienen el mismo peso.
\end{itemize}

\end{frame}

\subsection{Algoritmo de Kruskal}

\begin{frame}{Algoritmo de Kruskal}

\textbf{Procedimiento:}
\begin{enumerate}
    \item Ordenar todas las aristas del grafo en orden ascendente según su peso.
    \item Añadir la arista de menor peso que no forme un ciclo.
    \item Repetir hasta tener $n-1$ aristas.
\end{enumerate}

\textbf{Complejidad:}  
$O(E \log E)$ siendo $E$ la cantidad de aristas del grafo.

\end{frame}

\begin{frame}{Ejemplo (Kruskal)}
\centering
\includegraphics[scale=1]{}
\vspace{0.3cm}

{\small Selección en orden: $AB(1)\rightarrow CD(2)\rightarrow BC(3)$\\
Peso total: $1+2+3=6$}

\end{frame}

\subsection{Algoritmo de Prim}

\begin{frame}{Algoritmo de Prim}

\textbf{Procedimiento:}
\begin{enumerate}
    \item Seleccionar un vértice inicial arbitrario.
    \item Añadir la arista más corta que conecte un vértice del árbol actual con un nuevo vértice externo.
    \item Repetir hasta que todos los vértices estén en el árbol.
\end{enumerate}

\textbf{Complejidad:} $O(E \log V)$ usando estructuras eficientes (heap).

\end{frame}

\begin{frame}{Ejemplo (Prim)}

\centering
\begin{tikzpicture}[node distance=2cm, scale=0.8, every node/.style={scale=0.8}]
    \node[circle, draw] (A) at (0,3) {A};
    \node[circle, draw] (B) at (2,5) {B};
    \node[circle, draw] (C) at (4,3) {C};
    \node[circle, draw] (D) at (2,1) {D};

    \draw (A)--(B) node[midway,left] {1};
    \draw (B)--(C) node[midway,right] {3};
    \draw (C)--(D) node[midway,right] {2};
    \draw (D)--(A) node[midway,left] {4};
    \draw (A)--(C) node[midway,below] {5};
    \draw (B)--(D) node[midway,right] {6};
\end{tikzpicture}

\vspace{0.3cm}

{\small Comenzando en A:\\
$AB(1)\rightarrow BC(3)\rightarrow CD(2)$\\
Peso total: $1+3+2=6$}

\end{frame}

\subsection{Aplicaciones}

\begin{frame}{Aplicaciones prácticas del MST}

\begin{itemize}
    \item Diseño de redes de telecomunicaciones con costo mínimo.
    \item Construcción de caminos o carreteras de mínimo costo que conecten puntos específicos.
    \item Distribución eléctrica eficiente en una red de ciudades.
    \item Optimización logística y distribución de mercancía.
\end{itemize}

\end{frame}

\begin{frame}{Conclusión}

\begin{itemize}
    \item Los algoritmos de Kruskal y Prim permiten determinar eficazmente el Árbol de Expansión Mínima.
    \item Se utilizan comúnmente para optimizar costos en diferentes tipos de redes físicas y lógicas.
    \item Elegir entre ambos depende principalmente del tipo de grafo y estructura de datos utilizada.
\end{itemize}

\end{frame}

\end{document}
