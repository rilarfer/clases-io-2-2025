\documentclass{beamer}

\usepackage[utf8]{inputenc}
\usepackage[spanish]{babel}
\usepackage{amsmath}
\usepackage{amssymb}
\usepackage{tikz}
\usetikzlibrary{arrows.meta}
\usetheme{Goddard}
\hypersetup{colorlinks,allcolors=.,urlcolor=magenta}

\title{Investigación de Operaciones II}
\subtitle{Unidad 1: Análisis de Redes}
\author{Ricardo Jesús Largaespada Fernández}
\institute{Ingeniería de Sistemas, DACTIC, UNI}
\date{24 de Marzo, 2025}

\begin{document}

% ---------------------------------------------------------------
% TÍTULO Y AGENDA
% ---------------------------------------------------------------
\frame{\titlepage}

\begin{frame}{Agenda}
    \tableofcontents
\end{frame}

% ---------------------------------------------------------------
% SESIÓN 7
% ---------------------------------------------------------------
\section{Sesión 7}

% ---------------------------------------------------------------
% FRAME: Sesión 7 - Introducción
% ---------------------------------------------------------------
\begin{frame}{Sesión 7}
\textbf{Tema:}
\begin{enumerate}
    \item Problema de Flujo de Costo Mínimo: Enfoque Unificador.
\end{enumerate}

\textbf{Objetivo:}
\begin{itemize}
    \item Comprender la estructura general del problema de flujo de costo mínimo y su aplicación unificada a problemas de redes.
\end{itemize}
\end{frame}

% ---------------------------------------------------------------
% SUBSECCIÓN: Flujo de Costo Mínimo
% ---------------------------------------------------------------
\subsection{Problema de Flujo de Costo Mínimo}

% ---------------------------------------------------------------
% FRAME: Visión General
% ---------------------------------------------------------------
\begin{frame}{Visión General}
\textbf{Características Clave:}
\begin{itemize}
    \item Problema de programación lineal en redes
    \item Generaliza múltiples problemas clásicos:
    \begin{itemize}
        \item Ruta más corta
        \item Flujo máximo
        \item Transporte/Asignación
    \end{itemize}
    \item Resoluble con el \textbf{Método Símplex de Redes}
\end{itemize}

\begin{block}{Estructura Básica}
Minimizar: $\sum_{(i,j)\in A} c_{ij}x_{ij}$ \\
Sujeto a: \\
$\sum_{j} x_{ij} - \sum_{j} x_{ji} = b_i$ (Conservación de flujo) \\
$l_{ij} \leq x_{ij} \leq u_{ij}$ (Límites de capacidad)
\end{block}
\end{frame}

% ---------------------------------------------------------------
% FRAME: Relación con Otros Problemas
% ---------------------------------------------------------------
\begin{frame}{Mapa de Reducción}

\begin{center}
\begin{tikzpicture}[every node/.style={align=center}, node distance=2cm]
    % Nodos con más espacio
    \node (MCNF) at (0,4) {MCNF};
    \node (Transshipment) at (-3,2) {Transbordo};
    \node (MaxFlow) at (3,2) {Flujo Máximo};
    \node (Transportation) at (-3,0) {Transporte};
    \node (ShortestPath) at (0,0) {Camino Más Corto};
    \node (Assignment) at (-3,-2) {Asignación};

    % Flechas invertidas y con más espacio para etiquetas
    \draw[->] (Transshipment) -- node[below left] {($u_{ij} = \infty$)} (MCNF);
    \draw[->] (MaxFlow) -- node[midle right] {($c_{ij} = 0\ \forall (i,j) \in E$)\\ ($c_{ts} = -1$)} (MCNF);
    \draw[->] (Transportation) -- node[below left] {($b_i \neq 0$)} (Transshipment);
    \draw[->] (ShortestPath) -- node[midle right] {($b_i = 0\ \forall i \in T$)\\ ($b_s = 1, b_t = -1$)} (Transshipment);
    \draw[->] (Assignment) -- node[left] {($b_i = \pm1$)} (Transportation);
\end{tikzpicture}
\end{center}

\end{frame}

% ---------------------------------------------------------------
% FRAME: Método Símplex de Redes
% ---------------------------------------------------------------
\begin{frame}{Método Símplex de Redes}
\textbf{Ventajas sobre el Símplex Estándar:}
\begin{itemize}
    \item Explota la estructura de red para eficiencia computacional
    \item Trabaja con árboles de expansión en lugar de matrices completas
    \item Complejidad reducida: $O(n^2)$ vs $O(2^n)$ en casos prácticos
\end{itemize}

\textbf{Etapas del Algoritmo:}
\begin{enumerate}
    \item Encontrar solución básica factible inicial
    \item Identificar arco entrante (menor costo reducido)
    \item Determinar ciclo de actualización
    \item Ajustar flujos y actualizar árbol
    \item Repetir hasta optimalidad
\end{enumerate}
\end{frame}

% ---------------------------------------------------------------
% FRAME: Caso de Estudio - Nicaragua
% ---------------------------------------------------------------
\section{Paso a Paso: Algoritmo Símplex de Redes}

% ---------------------------------------------------------------
% FRAME: Definición del Problema
% ---------------------------------------------------------------
\begin{frame}{Definición del Problema Ejemplo}
\centering
\begin{tikzpicture}[scale=0.8,every node/.style={scale=0.8}]
    \node[circle,draw] (1) at (0,0) {1 \\ (100)};
    \node[circle,draw] (2) at (2,2) {2 \\ (0)};
    \node[circle,draw] (3) at (2,-2) {3 \\ (0)};
    \node[circle,draw] (4) at (4,2) {4 \\ (-60)};
    \node[circle,draw] (5) at (4,-2) {5 \\ (-40)};
    
    \draw[->] (1) to node[above left] {$(30, 5)$} (2);
    \draw[<-] (1) to node[below left] {$(40, 4)$} (3);
    \draw[->] (2) to node[above] {$(20, 3)$} (3);
    \draw[->] (2) to node[above] {$(50, 6)$} (4);
    \draw[->] (3) to node[below] {$(30, 5)$} (5);
    \draw[->] (2) to node[left] {$(∞, 2)$} (5);
    \draw[->] (3) to node[right] {$(∞, 3)$} (4);
\end{tikzpicture}

\begin{itemize}
    \item \textbf{Nodos:} 1(Fuente), 4-5(Destino), 2-3(Transbordo)
    \item \textbf{Arcos:} $(l_{ij}, u_{ij}, c_{ij})$ = (flujo mínimo, máximo, costo)
    \item \textbf{Objetivo:} Minimizar costo total cumpliendo demandas
\end{itemize}
\end{frame}

% ---------------------------------------------------------------
% FRAME: Paso 1 - Árbol de Expansión Inicial
% ---------------------------------------------------------------
\begin{frame}{Paso 1: Árbol de Expansión Inicial}
\centering
\begin{tikzpicture}[scale=0.7]
    \node[circle,draw] (1) at (0,0) {1};
    \node[circle,draw] (2) at (2,2) {2};
    \node[circle,draw] (3) at (2,-2) {3};
    \node[circle,draw] (4) at (4,2) {4};
    \node[circle,draw] (5) at (4,-2) {5};
    
    \draw[thick,red] (1) -- (2) node[midway,above left] {30/5};
    \draw[thick,red] (1) -- (3) node[midway,below left] {40/4};
    \draw[thick,red] (2) -- (4) node[midway,above] {30/6};
    \draw[thick,red] (3) -- (5) node[midway,below] {30/5};
\end{tikzpicture}

\begin{block}{Solución Básica Factible Inicial}
\begin{itemize}
    \item Seleccionamos arcos: (1,2), (1,3), (2,4), (3,5)
    \item Flujos: $x_{12}=30$, $x_{13}=40$, $x_{24}=30$, $x_{35}=30$
    \item Costo inicial: $30×5 + 40×4 + 30×6 + 30×5 = 730$
\end{itemize}
\end{block}
\end{frame}

% ---------------------------------------------------------------
% FRAME: Paso 2 - Cálculo de Potenciales
% ---------------------------------------------------------------
\begin{frame}{Paso 2: Cálculo de Potenciales ($\pi_i$)}
\begin{columns}
\column{0.5\textwidth}
\centering
\begin{tikzpicture}[scale=0.6]
    \node[circle,draw] (1) at (0,0) {1 \\ $\pi_1=0$};
    \node[circle,draw] (2) at (2,2) {2 \\ $\pi_2=5$};
    \node[circle,draw] (3) at (2,-2) {3 \\ $\pi_3=4$};
    \node[circle,draw] (4) at (4,2) {4 \\ $\pi_4=11$};
    \node[circle,draw] (5) at (4,-2) {5 \\ $\pi_5=9$};
    
    \draw[thick,red] (1) -- (2);
    \draw[thick,red] (1) -- (3);
    \draw[thick,red] (2) -- (4);
    \draw[thick,red] (3) -- (5);
\end{tikzpicture}

\column{0.5\textwidth}
\begin{align*}
    \pi_1 &= 0 \quad (\text{fijado}) \\
    \pi_2 - \pi_1 &= 5 \implies \pi_2 = 5 \\
    \pi_3 - \pi_1 &= 4 \implies \pi_3 = 4 \\
    \pi_4 - \pi_2 &= 6 \implies \pi_4 = 11 \\
    \pi_5 - \pi_3 &= 5 \implies \pi_5 = 9
\end{align*}
\end{columns}

\begin{block}{Fórmula de Potenciales}
Para cada arco básico $(i,j)$: \\
$\pi_j - \pi_i = c_{ij}$
\end{block}
\end{frame}

% ---------------------------------------------------------------
% FRAME: Paso 3 - Costos Reducidos
% ---------------------------------------------------------------
\begin{frame}{Paso 3: Cálculo de Costos Reducidos}
\begin{tabular}{l|c|c}
Arco no básico & Cálculo & Valor \\ \hline
(2,3) & $3 - \pi_2 + \pi_3 = 3 -5 +4$ & 2 \\
(2,5) & $2 -5 +9$ & 6 \\
(3,4) & $3 -4 +11$ & 10 \\
(3,5) & \textcolor{green}{5 -4 +9} & \textcolor{green}{10} \\
\end{tabular}

\begin{block}{Fórmula de Costo Reducido}
$\bar{c}_{ij} = c_{ij} - \pi_i + \pi_j$ \\
\small
Si $\bar{c}_{ij} < 0$: arco candidato a entrar
\end{block}

\vspace{1em}
\textbf{Conclusión:} No hay costos reducidos negativos $\rightarrow$ ¿Solución óptima? \alert{¡No!} Falta verificar capacidad
\end{frame}

% ---------------------------------------------------------------
% FRAME: Paso 4 - Análisis de Capacidad
% ---------------------------------------------------------------
\begin{frame}{Paso 4: Análisis de Capacidad}
\begin{columns}
\column{0.6\textwidth}
\centering
\begin{tikzpicture}[scale=0.6]
    \node[circle,draw] (1) at (0,0) {1};
    \node[circle,draw] (2) at (2,2) {2};
    \node[circle,draw] (3) at (2,-2) {3};
    \node[circle,draw] (4) at (4,2) {4};
    \node[circle,draw] (5) at (4,-2) {5};
    
    \draw[->,dashed] (2) to [bend right] node[midway,above] {+θ} (5);
    \draw[->,dashed] (5) to [bend right] node[midway,below] {-θ} (3);
    \draw[->,dashed] (3) to [bend right] node[midway,below] {+θ} (1);
    \draw[->,dashed] (1) to [bend right] node[midway,above] {-θ} (2);
\end{tikzpicture}

\column{0.4\textwidth}
\textbf{Ciclo de Ajuste:}
\begin{itemize}
    \item Entra: (2,5)
    \item Sale: (1,2)
    \item θ máxima: 30
\end{itemize}
\end{columns}

\begin{block}{Actualización de Flujos}
\begin{align*}
    x_{25} &= 0 + 30 = 30 \\
    x_{12} &= 30 - 30 = 0 \\
    x_{13} &= 40 + 30 = 70 \\
    x_{35} &= 30 - 30 = 0
\end{align*}
\end{block}
\end{frame}

% ---------------------------------------------------------------
% FRAME: Paso 5 - Nueva Solución Óptima
% ---------------------------------------------------------------
\begin{frame}{Paso 5: Nueva Solución Óptima}
\centering
\begin{tikzpicture}[scale=0.7]
    \node[circle,draw] (1) at (0,0) {1};
    \node[circle,draw] (2) at (2,2) {2};
    \node[circle,draw] (3) at (2,-2) {3};
    \node[circle,draw] (4) at (4,2) {4};
    \node[circle,draw] (5) at (4,-2) {5};
    
    \draw[thick,red] (1) -- (3) node[midway,left] {70/4};
    \draw[thick,red] (2) -- (4) node[midway,above] {30/6};
    \draw[thick,red] (2) -- (5) node[midway,right] {30/2};
    \draw[thick,red] (3) -- (4) node[midway,below] {30/3};
\end{tikzpicture}

\begin{block}{Costo Actualizado}
\begin{itemize}
    \item Flujos: $x_{13}=70$, $x_{24}=30$, $x_{25}=30$, $x_{34}=30$
    \item Nuevo costo: $70×4 + 30×6 + 30×2 + 30×3 = 670$ (\textcolor{green}{↓60})
\end{itemize}
\end{block}

\begin{alertblock}{Verificación Final}
No hay más costos reducidos negativos $\rightarrow$ \textbf{Solución Óptima}
\end{alertblock}
\end{frame}

% ---------------------------------------------------------------
% FRAME: Conclusión
% ---------------------------------------------------------------
\begin{frame}{Conclusión}
\begin{itemize}
    \item El flujo de costo mínimo integra múltiples problemas clásicos
    \item El símplex de redes provee ventajas computacionales decisivas
    \item Aplicaciones nacionales demuestran su utilidad práctica
    \item Herramienta fundamental para la toma de decisiones en redes
\end{itemize}

\begin{alertblock}{Próxima Sesión}
Implementación computacional con Python NetworkX y casos reales
\end{alertblock}
\end{frame}

\end{document}