\documentclass{beamer}

\usepackage[utf8]{inputenc}
\usepackage[spanish]{babel}
\usepackage{amsmath}
\usepackage{amssymb}
\usepackage{tikz}
\usetheme{Goddard}
\hypersetup{colorlinks,allcolors=.,urlcolor=magenta}

\title{Investigación de Operaciones II}
\subtitle{Unidad 1: Análisis de Redes}
\author{Ricardo Jesús Largaespada Fernández}
\institute{Ingeniería de Sistemas, DACTIC, UNI}
\date{15 de Marzo, 2025}

\begin{document}

% ---------------------------------------------------------------
% TÍTULO Y AGENDA
% ---------------------------------------------------------------
\frame{\titlepage}

\begin{frame}{Agenda}
    \tableofcontents
\end{frame}

% ---------------------------------------------------------------
% SESIÓN 5
% ---------------------------------------------------------------
\section{Sesión 5}

% ---------------------------------------------------------------
% FRAME: Sesión 5 - Introducción
% ---------------------------------------------------------------
\begin{frame}{Sesión 5}
\textbf{Tema:}
\begin{enumerate}
    \item Método de Flujo Máximo (Ford-Fulkerson).
\end{enumerate}

\textbf{Objetivo:}
\begin{itemize}
    \item El estudiante aplicará el algoritmo de Ford-Fulkerson para resolver problemas de flujo máximo en redes dirigidas con capacidad.
\end{itemize}
\end{frame}

% ---------------------------------------------------------------
% SUBSECCIÓN: Método de Flujo Máximo
% ---------------------------------------------------------------
\subsection{Método de Flujo Máximo}

% ---------------------------------------------------------------
% FRAME: Definición y Aplicaciones
% ---------------------------------------------------------------
\begin{frame}{Definición y Aplicaciones}
\textbf{El problema del flujo máximo} consiste en determinar la \textbf{mayor cantidad de flujo} que se puede enviar desde un \textbf{nodo origen} a un \textbf{nodo destino} en una red, sin superar las \textbf{capacidades} de los arcos.

\medskip
\textbf{Aplicaciones comunes:}
\begin{itemize}
    \item Distribución de productos desde fábricas hasta clientes.
    \item Suministro de materias primas de proveedores a plantas industriales.
    \item Sistemas de tuberías para transportar petróleo o agua.
    \item Redes de transporte (maximizar el flujo de vehículos).
\end{itemize}
\end{frame}

% ---------------------------------------------------------------
% FRAME: Algoritmo Ford-Fulkerson
% ---------------------------------------------------------------
\begin{frame}{Algoritmo Ford-Fulkerson}
\textbf{Ideas principales:}
\begin{enumerate}
    \item \textbf{Caminos de aumento}: Se busca un camino desde la fuente al sumidero donde se pueda incrementar el flujo sin superar las capacidades.
    \item \textbf{Red residual}: La capacidad disponible en cada arco luego de considerar el flujo ya asignado.
    \item \textbf{Iterativo}: Inicia con flujo $f(u,v)=0$ para todo par de nodos y, en cada paso, se incrementa el flujo hasta no encontrar más caminos de aumento.
\end{enumerate}

\medskip
\textbf{Pseudocódigo:}
\begin{block}{Ford-Fulkerson\_method$(G,s,t)$}
\textbf{inicializar} flujo $f \leftarrow 0$;\\
\textbf{while} (exista un camino de aumento $p$ en la red residual) \textbf{do}\\
\quad aumentar el flujo $f$ a lo largo de $p$;\\
\textbf{return} $f$;
\end{block}

\end{frame}

% ---------------------------------------------------------------
% FRAME: Problema Propuesto (sin solución)
% ---------------------------------------------------------------
\begin{frame}{Problema Propuesto}
\textbf{Enunciado:}

Considere la siguiente red de flujo con nodo origen $O$ y nodo destino $T$:
\[
\begin{aligned}
&\text{Capacidades:}\\
& O \rightarrow A = 5,\quad O \rightarrow B = 7, \\
& A \rightarrow C = 4,\quad A \rightarrow B = 2, \\
& B \rightarrow C = 3,\quad B \rightarrow D = 5, \\
& C \rightarrow T = 6,\quad D \rightarrow T = 3.
\end{aligned}
\]

Determine, utilizando el \textbf{algoritmo de Ford-Fulkerson}, el \emph{flujo máximo} que puede enviarse desde $O$ hasta $T$. 

\medskip
(\textit{Se sugiere identificar los posibles caminos de aumento y actualizar sucesivamente las capacidades en la red residual.})
\end{frame}

% ---------------------------------------------------------------
% FRAME: Solución (Paso a Paso)
% ---------------------------------------------------------------
\begin{frame}{Solución (Paso a Paso)}
\textbf{Idea para la resolución:}
\begin{enumerate}
    \item Inicializar $f(u,v) = 0$ para todos los arcos.
    \item Identificar \textit{caminos de aumento} (por ejemplo, $O \rightarrow B \rightarrow D \rightarrow T$).
    \item Calcular el \textit{cuello de botella} en cada camino y \textit{aumentar} el flujo en consecuencia.
    \item Actualizar la \textit{red residual}: 
    \[
    \text{capacidad residual} = c(u,v) - f(u,v).
    \]
    \item Repetir mientras existan caminos de aumento disponibles.
\end{enumerate}

\textbf{Resultado final:}  
Se determina el \textbf{flujo máximo} (suma del flujo que sale de $O$, o bien, el que llega a $T$) tras agotar los caminos de aumento.

\end{frame}

\begin{frame}{Ejemplo de Flujo Máximo: Red Inicial}
\centering
\begin{tikzpicture}[>=latex,scale=1.1,every node/.style={scale=1}]
    % Nodos
    \node[circle,draw,thick] (1) at (2,0) {1};
    \node[circle,draw,thick] (2) at (3,-2) {2};
    \node[circle,draw,thick] (3) at (8,0) {3};
    \node[circle,draw,thick] (4) at (3,2) {4};
    \node[circle,draw,thick] (5) at (6,2) {5};

    % Arcos (con capacidad en rojo)
    \draw (1) -- (2) node[midway,below left,xshift=-1mm,yshift=1mm,red]{(20,0)};
    \draw (1) -- (3) node[midway,above left ,yshift=-1mm,red]{(30,0)};
    \draw (1) -- (4) node[midway,above left,yshift=-1mm,red]{(10,0)};
    \draw (2) -- (3) node[midway,below,yshift=-1mm,red]{(40,0)};
\draw[bend left] (2) to node[midway,above left,xshift=1mm,yshift=1mm,red]{(30,0)} (5);    \draw (3) -- (4) node[midway,aboveleft,xshift=-1mm,red]{(10,5)};
    \draw (3) -- (5) node[midway,right,xshift=1mm,red]{(20,0)};
    \draw (4) -- (5) node[midway,above,yshift=1mm,red]{(20,0)};
\end{tikzpicture}

\vspace{0.5em}
\textbf{Capacidades iniciales} (en rojo) y flujo actual (0). El objetivo es encontrar el \emph{máximo flujo} entre el nodo 1 (origen) y el nodo 5 (destino).
\end{frame}

% -------------------------------------------------------------------------------------
% FRAME 2: ITERACIÓN 1
% -------------------------------------------------------------------------------------
\begin{frame}{Iteración 1}
\small
\textbf{Pasos clave:}
\begin{enumerate}
    \item \(\displaystyle j=1,\; [a_1] = [\infty,\,-]\)
    \item Se obtiene \(S(j) = S(1) = \{2,3,4\}\,\neq \varnothing\). 
    \item Se calcula \(\max\{f_{12}, f_{13}, f_{14}\} = \max\{\,20, 30, 10\,\} = 30 \implies [a_{3},1] = [30,1]\).
    \item Nuevamente, \(S(j) = \{4,5\} \neq \varnothing\).
    \item \(\max\{f_{34}, f_{35}\} = \max\{10, 20\}=20 \implies [a_{5},3] = [20,3]\).
    \item \textbf{Trayectoria:} \(5 \to [20,3] \to 3 \to [30,1] \to 1\).
    \item Flujo enviado: \(f_1 = \min\{\,a_{3}, a_{5}\} = \min\{30,20\} = 20\).
    \item \textbf{Actualización de capacidades residuales:}\\
          \((1\to 3)\) pasa de \((30,0)\) a \((10,20)\), etc.
\end{enumerate}

\textbf{Resultado de Iteración 1:} Se inyectan 20 unidades de flujo a través de la ruta \((5\rightarrow 3\rightarrow 1)\) (en sentido inverso para la actualización interna).
\end{frame}

% -------------------------------------------------------------------------------------
% FRAME 3: ITERACIÓN 2
% -------------------------------------------------------------------------------------
\begin{frame}{Iteración 2}
\small
\textbf{Resumen de Pasos:}
\begin{enumerate}
    \item \(\displaystyle j=1,\; S(1)=\{2,3,4\}\).
    \item \(\max\{f_{12}, f_{13}, f_{14}\} = \max\{20,10,10\} = 20 \implies [a_1] = [20,1]\).
    \item \(\displaystyle j=2,\; S(2)=\{3,5\}\), \(\max\{f_{23}, f_{25}\} = \max\{40,30\} = 40 \implies [a_2] = [40,2]\).
    \item \(\displaystyle j=3,\; S(3)=\{4\}\), \(\max\{f_{34}\} = 10 \implies [a_3]=[10,3]\).
    \item \(\displaystyle j=4,\; S(4)=\{5\}\), \(\max\{f_{45}\} = 20 \implies [a_4]=[20,4]\).
    \item \textbf{Trayectoria:} \(5 \to [20,4] \to 4 \to [10,3] \to 3 \to [40,2] \to 2 \to [20,1] \to 1\).
    \item Flujo enviado: \(f_2 = \min\{20,40,10,20\} = 10\).
    \item Actualización de capacidades residuales en arcos: 
          \((2\to5)\) de \((30,0)\) a \((20,10)\), etc.
\end{enumerate}

\textbf{Resultado de Iteración 2:} Se añaden 10 unidades de flujo por la nueva ruta.
\end{frame}

% -------------------------------------------------------------------------------------
% FRAME 4: ITERACIÓN 3
% -------------------------------------------------------------------------------------
\begin{frame}{Iteración 3}
\small
\begin{enumerate}
    \item \(\displaystyle j=1,\; S(1) = \{2,3,4\}\). 
          \(\max\{f_{12},f_{13},f_{14}\} = \max\{10,10,10\}=10\implies[a_1] = [10,1]\).
    \item \(\displaystyle j=2,\; S(2) = \{3,5\}\). 
          \(\max\{f_{23}, f_{25}\} = \max\{30,30\}=30\implies[a_2] = [30,2]\).
    \item \(\displaystyle j=3 \to S(3)= \varnothing\). 
          No se puede ir al nodo 4 ni al 5, se hace \emph{backtrack}.
    \item Se elimina la etiqueta \([30,2]\) y se retrasa a \(j=2\). 
    \item Nuevamente, \(\max\{f_{25}\}=30 \implies[a_2] = [30,2]\).
    \item \textbf{Trayectoria final}: \(5 \to [30,2] \to 2 \to [10,1] \to 1\).
    \item \(\displaystyle f_3 = \min\{10,30\}=10\).
    \item Se actualizan las capacidades residuales.
\end{enumerate}

\textbf{Resultado de Iteración 3:} Se suman 10 unidades más de flujo.
\end{frame}

% -------------------------------------------------------------------------------------
% FRAME 5: ITERACIÓN 4
% -------------------------------------------------------------------------------------
\begin{frame}{Iteración 4}
\small
\begin{enumerate}
    \item \(\displaystyle j=1,\; S(1)=\{3,4\}\neq \varnothing\). 
          \(\max\{f_{13}, f_{14}\} = \max\{10,10\}=10\implies[a_3,1]=[10,1]\).
    \item \(\displaystyle j=2,\; S(3)=\{2\}\). 
          \(\max\{f_{32}\}=10\implies [a_2,3]=[10,3]\).
    \item \(\displaystyle j=3,\; S(2)=\{5\}\). 
          \(\max\{f_{25}\}=20\implies [a_5,2]=[20,2]\).
    \item \textbf{Trayectoria}: \(5 \to [20,2] \to 2 \to [10,3] \to 3 \to [10,1] \to 1\).
    \item \(\displaystyle f_4 = \min\{10,10,20\} = 10\).
    \item Actualización de residuales: 
          \((4\to5)\), etc.
\end{enumerate}

\textbf{Resultado de Iteración 4:} Se añaden 10 unidades más.
\end{frame}

% -------------------------------------------------------------------------------------
% FRAME 6: ITERACIÓN 5
% -------------------------------------------------------------------------------------
\begin{frame}{Iteración 5}
\small
\begin{enumerate}
    \item \(\displaystyle j=1,\; S(1)=\{4\}\). 
          \(\max\{f_{14}\}=10 \implies [a_4,1]=[10,1]\).
    \item \(\displaystyle j=4,\; S(4)=\{3,5\}\). 
          \(\max\{f_{43}, f_{45}\}=\max\{15,10\}=15\implies[a_3,4]=[15,4]\).
    \item \(\displaystyle j=3 \to S(3)= \varnothing \). \emph{Retroceso.}
          Se elimina \([15,4]\).  
    \item Regresamos a \(j=4,\;\max\{f_{45}\}=10 \implies [a_5,4]=[10,4]\).
    \item \textbf{Trayectoria}: \(5\to[10,4]\to4\to[10,1]\to1\).
    \item \(\displaystyle f_5 = \min\{10,10\}=10\).
    \item Se actualizan capacidades residuales.
\end{enumerate}

\textbf{Resultado de Iteración 5:} Sumamos 10 unidades finales.
\end{frame}

% -------------------------------------------------------------------------------------
% FRAME 7: RESUMEN DEL FLUJO
% -------------------------------------------------------------------------------------
\begin{frame}{Resumen de Flujo Final}
\small
\begin{tabular}{|c|c|c|c|}
\hline
\textbf{Arco} & \textbf{Cap. Inicial -- Cap. Final} & \textbf{Flujo Total} & \textbf{Dirección} \\
\hline
$(1,2)$ & $(20,0)\rightarrow(20,-20)$ & 20 & $1\rightarrow 2$ \\
$(1,3)$ & $(30,0)\rightarrow(30,-30)$ & 30 & $1\rightarrow 3$ \\
$(1,4)$ & $(10,0)\rightarrow(10,-10)$ & 10 & $1\rightarrow 4$ \\
$(2,3)$ & $(40,0)\rightarrow(40,0)$ & 0  & -- \\
$(2,5)$ & $(30,0)\rightarrow(20,-20)$ & 20 & $2\rightarrow 5$ \\
$(3,4)$ & $(10,5)\rightarrow(10,-15)$ & 10 & $3\rightarrow 4$ \\
$(3,5)$ & $(20,0)\rightarrow(20,-20)$ & 20 & $3\rightarrow 5$ \\
$(4,5)$ & $(20,0)\rightarrow(30,-20)$ & 20 & $4\rightarrow 5$ \\
\hline
\end{tabular}

\vspace{1em}
\textbf{Flujo Máximo Total = 20 + 20 + 20 = 60.}

(Dependiendo de la ruta final y ajustes, la suma o la verificación del flujo en el nodo fuente/destino corroboran este resultado.)
\end{frame}
\end{document}
