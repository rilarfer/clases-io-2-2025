\documentclass{beamer}

\usepackage[utf8]{inputenc}
\usepackage[spanish]{babel}
\usepackage{amsmath}
\usepackage{amssymb}
\usepackage{tikz}
\usepackage{pgfplots}
\pgfplotsset{compat=1.18}

\usetikzlibrary{arrows,calc,positioning}
%\usetheme{Goddard}
\usetheme{Madrid}
\hypersetup{colorlinks,allcolors=.,urlcolor=magenta}
\usepackage{listings}
\lstset{
    language=Matlab,
    basicstyle=\ttfamily\footnotesize,
    breaklines=true,
    frame=single,
    columns=flexible
}

\title{Investigación de Operaciones II}
\subtitle{Unidad 3: Fundamentos de la Teoría de Colas}
\author[RL]{Ricardo Jesús Largaespada Fernández}
\institute[UNI]{Ingeniería de Sistemas, DACTIC, UNI}
\date{19 de Mayo, 2025}

\begin{document}

\begin{frame}
  \titlepage
\end{frame}

% --- Introducci\'on ---

\section*{Introducci\'on}

\begin{frame}{Motivación: Más allá de los modelos exponenciales}
  \begin{itemize}
    \item La mayoría de los modelos clásicos de teoría de colas suponen que:
    \begin{itemize}
      \item los tiempos entre llegadas siguen una distribución exponencial, y \pause
      \item los tiempos de servicio también son exponenciales. \pause
    \end{itemize}

    \item No obstante, en muchos sistemas reales estas suposiciones no se cumplen: \pause
    \begin{itemize}
      \item Las llegadas pueden estar programadas o controladas. \pause
      \item Los tiempos de servicio pueden ser constantes o tener baja variabilidad. \pause
    \end{itemize}

    \item En estos casos, es necesario utilizar modelos más generales que permitan otras distribuciones. \pause

    \item Estos modelos permiten un análisis más realista del desempeño del sistema y una mejor toma de decisiones operacionales.
  \end{itemize}
\end{frame}


% --- M/G/1 ---

\section*{Modelo M/G/1}

\begin{frame}{Modelo \texorpdfstring{$M/G/1$}{M/G/1}}
  \begin{itemize}
    \item El modelo $M/G/1$ describe un sistema de colas con las siguientes características: \pause
    \begin{itemize}
      \item Llegadas al sistema según un proceso de Poisson con tasa \( \lambda \). \pause
      \item Un único servidor. \pause
      \item Tiempos de servicio con distribución general (no necesariamente exponencial). \pause
    \end{itemize}

    \item A diferencia del modelo \( M/M/1 \), aquí no se impone una forma específica para la distribución del servicio. \pause

    \item Es suficiente conocer dos parámetros fundamentales del tiempo de servicio: \pause
    \begin{itemize}
      \item La media: \( \mathbb{E}[S] = 1/\mu \). \pause
      \item La varianza: \( \sigma^2 = \mathrm{Var}(S) \). \pause
    \end{itemize}
    
    \item La generalidad del modelo permite representar con mayor fidelidad diversos entornos reales de atención al cliente.
  \end{itemize}
\end{frame}

\begin{frame}{Resultados del modelo \texorpdfstring{$M/G/1$}{M/G/1} (I)}
  Bajo condiciones de estabilidad (\( \rho = \lambda/\mu < 1 \)), se tiene:

  \pause
  \[
  L_q = \frac{\lambda^2 \sigma^2 + \rho^2}{2(1 - \rho)}
  \]

  \pause
  Esta expresión se conoce como la \textbf{fórmula de Pollaczek–Khinchine}. Involucra:
  \begin{itemize}
    \item \( \sigma^2 \): varianza del tiempo de servicio.
    \item \( \rho = \lambda/\mu \): utilización del sistema.
  \end{itemize}
  \pause
  Refleja cómo la variabilidad y la carga del sistema afectan la longitud promedio de la cola.
\end{frame}

\begin{frame}{Resultados del modelo \texorpdfstring{$M/G/1$}{M/G/1} (II)}
  A partir de \( L_q \), se obtienen las métricas restantes:

  \pause
  \begin{align*}
    L &= L_q + \rho \\[1ex] \pause
    W_q &= \frac{L_q}{\lambda} \\[1ex] \pause
    W &= W_q + \frac{1}{\mu}
  \end{align*}

  \pause
  \begin{itemize}
    \item \( L \): número promedio de clientes en el sistema.
    \item \( W_q \): tiempo promedio en espera.
    \item \( W \): tiempo total promedio en el sistema.
  \end{itemize}
\end{frame}

\begin{frame}{Impacto de la varianza en el modelo \texorpdfstring{$M/G/1$}{M/G/1}}
  \begin{itemize}
    \item La varianza \( \sigma^2 \) del tiempo de servicio es un parámetro crítico en el modelo \( M/G/1 \). \pause

    \item A mayor \( \sigma^2 \), mayor incertidumbre en los tiempos de atención, lo cual incrementa la congestión del sistema. \pause

    \item En la fórmula de Pollaczek--Khinchine:
    \[
    L_q = \frac{\lambda^2 \sigma^2 + \rho^2}{2(1 - \rho)}
    \]
    se observa que \( L_q \) crece linealmente con \( \sigma^2 \). \pause

    \item Caso límite: si \( \sigma^2 = 0 \) (tiempo de servicio constante), se obtiene el modelo determinista \( M/D/1 \), donde la cola es más corta. \pause

    \item Por tanto, reducir la variabilidad mejora el rendimiento del sistema sin necesidad de aumentar servidores o modificar tasas.
  \end{itemize}
\end{frame}

\begin{frame}{Efecto de la varianza en \texorpdfstring{$L_q$}{Lq}}

  \textbf{Supuestos:}
  \begin{itemize}
    \item Tasa de llegada: \( \lambda = 4 \) clientes/hora
    \item Tasa de servicio: \( \mu = 5 \) clientes/hora
    \item Utilización: \( \rho = \lambda / \mu = 0.8 \)
  \end{itemize}

  \vspace{0.5em}
  \begin{center}
    \begin{tikzpicture}
      \begin{axis}[
        width=8cm, height=4cm,
        xlabel={$\sigma^2$ (varianza del tiempo de servicio)},
        ylabel={$L_q$ (clientes en cola)},
        grid=major,
        axis lines=left,
        domain=0:1,
        samples=100,
        thick,
        enlargelimits=true,
        legend style={at={(0.97,0.97)}, anchor=north east, font=\small}
      ]
        \addplot[blue] expression{(4^2 * x + 0.8^2)/(2*(1 - 0.8))};
        \addlegendentry{$L_q = \frac{\lambda^2 \sigma^2 + \rho^2}{2(1 - \rho)}$}
      \end{axis}
    \end{tikzpicture}
  \end{center}

  \pause
  \begin{itemize}
    \item \( \sigma^2 = 0 \): modelo determinista \( M/D/1 \)
    \item \( \sigma^2 = 0.04 \): modelo exponencial \( M/M/1 \)
    \item A medida que \( \sigma^2 \) crece, \( L_q \) también aumenta: más variabilidad → más congestión
  \end{itemize}
\end{frame}

% --- Modelo M/D/1 ---

\section*{Modelo M/D/1}

\begin{frame}{Modelo \texorpdfstring{$M/D/1$}{M/D/1}}
  \begin{itemize}
    \item Supone llegadas al sistema según un proceso de Poisson con tasa \( \lambda \). \pause

    \item El tiempo de servicio es constante: todos los clientes tardan exactamente \( 1/\mu \) en ser atendidos. \pause

    \item Es un caso particular del modelo \( M/G/1 \) con varianza \( \sigma^2 = 0 \). \pause

    \item La fórmula de espera en cola es:
    \[
    L_q = \frac{\rho^2}{2(1 - \rho)}
    \] \pause

    \item Comparado con el modelo \( M/M/1 \), este valor es exactamente la mitad:
    \[
    L_q^{(M/D/1)} = \frac{1}{2} L_q^{(M/M/1)}
    \]
  \end{itemize}
\end{frame}

% --- Modelo M/E_k/1 ---

\section*{Modelo M/E$_k$/1}

\begin{frame}{Modelo \texorpdfstring{$M/E_k/1$}{M/Ek/1}}
  \begin{itemize}
    \item En este modelo, los tiempos de servicio siguen una distribución \textbf{Erlang} de orden \( k \). \pause

    \item Se interpreta como la suma de \( k \) fases exponenciales con parámetro \( k\mu \), es decir:
    \[
    S = \sum_{i=1}^{k} \text{Exp}(k\mu) \quad \Rightarrow \quad \mathbb{E}[S] = \frac{1}{\mu}, \quad \sigma = \frac{1}{\sqrt{k} \mu}
    \] \pause

    \item Este modelo permite ajustar la variabilidad del servicio:
    \begin{itemize}
      \item \( k = 1 \): se obtiene la exponencial → modelo \( M/M/1 \) \pause
      \item \( k \to \infty \): se aproxima a servicio constante → modelo \( M/D/1 \) \pause
    \end{itemize}

    \item Cuanto mayor es \( k \), menor es la variabilidad. \pause

    \item El modelo \( M/E_k/1 \) es útil para representar servicios moderadamente regulares, como estaciones automáticas o procesos semiautomatizados.
  \end{itemize}
\end{frame}

\begin{frame}{Resultados M/E$_k$/1}
\begin{align*}
L_q &= \frac{1 + k}{2k} \cdot \frac{\lambda^2}{\mu(\mu - \lambda)}, \\
W_q &= \frac{1 + k}{2k} \cdot \frac{\lambda}{\mu(\mu - \lambda)}.
\end{align*}
\[
W = W_q + \frac{1}{\mu}, \quad L = \lambda W
\]
\end{frame}

% --- Sin entradas de Poisson ---

\begin{frame}{Modelos sin entradas de Poisson}
  \begin{itemize}
    \item Hasta ahora, hemos supuesto que las llegadas al sistema siguen un proceso de Poisson (interarribos exponenciales). \pause

    \item Sin embargo, en muchos contextos reales:
    \begin{itemize}
      \item Las llegadas están programadas (p. ej., producción en línea). \pause
      \item Ocurre control o regulación externa del flujo. \pause
    \end{itemize}

    \item En estos casos, se rompe el supuesto de aleatoriedad en los interarribos. \pause

    \item Esto da lugar a modelos donde la distribución de llegadas es \textbf{determinista} o \textbf{general}, pero el servicio sigue siendo exponencial. \pause

    \item Ejemplos representativos:
    \begin{itemize}
      \item \( D/M/1 \): llegadas deterministas \pause
      \item \( E_k/M/s \): llegadas Erlang \pause
      \item \( G/M/s \): llegadas con distribución general
    \end{itemize}
  \end{itemize}
\end{frame}

\begin{frame}{Modelos con llegadas no Poisson}
  \begin{itemize}
    \item \textbf{\( G/M/s \)}: \pause
    \begin{itemize}
      \item Llegadas con distribución general. \pause
      \item Servicio exponencial con \( s \) servidores. \pause
      \item Modelo flexible pero difícil de resolver analíticamente. \pause
    \end{itemize}

    \item \textbf{\( D/M/1 \)}: \pause
    \begin{itemize}
      \item Llegadas ocurren a intervalos fijos y regulares. \pause
      \item Servicio exponencial. \pause
      \item Ideal para sistemas automatizados con producción sincronizada. \pause
    \end{itemize}

    \item \textbf{\( E_k/M/s \)}: \pause
    \begin{itemize}
      \item Los tiempos entre llegadas siguen una distribución Erlang. \pause
      \item Permite controlar la variabilidad mediante el parámetro \( k \). \pause
      \item Se comporta como un punto intermedio entre \( D/M/s \) y \( M/M/s \).
    \end{itemize}
  \end{itemize}
\end{frame}

% --- Otros modelos ---

\begin{frame}{Modelos con distribuciones complejas}
  \begin{itemize}
    \item \textbf{Distribución hiperexponencial:} \pause
    \begin{itemize}
      \item Es una combinación probabilística de dos o más distribuciones exponenciales con diferentes tasas. \pause
      \item Tiene mayor desviación estándar que una exponencial individual. \pause
      \item Útil para modelar escenarios con alta variabilidad: algunas tareas son rápidas, otras muy lentas. \pause
    \end{itemize}

    \item \textbf{Distribuciones tipo fase:} \pause
    \begin{itemize}
      \item Generalizan la Erlang: combinan varias fases exponenciales en serie, en paralelo o de forma mixta. \pause
      \item Pueden ajustarse con alta precisión a datos empíricos. \pause
      \item Se modelan mediante cadenas de Markov de tiempo continuo. \pause
      \item Incluyen como casos particulares: exponencial, Erlang, e hiperexponencial. \pause
    \end{itemize}

    \item Estos modelos permiten representar servicios complejos o variables sin renunciar al marco de Markov.
  \end{itemize}
\end{frame}

\end{document}
