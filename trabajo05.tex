\documentclass{article}

% Matemática y símbolos
\usepackage{amsmath}
\usepackage{amssymb}

% Márgenes y geometría de la página
\usepackage{geometry}
\geometry{a4paper, margin=1in, bottom=0.5cm}

% Colores y cajas
\usepackage{xcolor}
\usepackage{mdframed}

% Gráficos y diagramas
\usepackage{tikz}
\usetikzlibrary{positioning}  % Para posicionamiento relativo en TikZ

% Tablas elegantes y columnas personalizadas
\usepackage{booktabs}
\usepackage{array}
\usepackage{float}

% Imágenes
\usepackage{graphicx}

% Estilo de párrafo
\usepackage{parskip}

% Idioma
\usepackage[spanish]{babel}
\usepackage[utf8]{inputenc}  % Si no usas UTF-8 nativo

% Captions
\usepackage{caption}

% Información del documento
\title{Estado Estable | Estados Absorbentes}
\author{Ricardo Largaespada}
\date{29 de abril 2025}

% Entorno personalizado para problemas
\newmdenv[
  backgroundcolor=blue!5,
  linecolor=blue,
  linewidth=1pt,
  roundcorner=5pt,
  skipabove=\baselineskip,
  skipbelow=\baselineskip
]{problem}


\begin{document}

\maketitle

\vspace{-.7cm}
\begin{problem}
\textbf{Problema 1 | Modelo de infracciones de tránsito}

José es un conductor que acumula infracciones según las normas de tránsito vigentes en Nicaragua. Estas infracciones se clasifican en dos tipos:

\begin{itemize}
    \item \textbf{Multa amarilla}: se trata de una infracción leve.
    \item \textbf{Multa roja}: se trata de una infracción grave que conlleva suspensión inmediata.
\end{itemize}

El sistema de sanciones funciona de la siguiente manera:

\begin{itemize}
    \item Al recibir una \textbf{multa roja}, José es suspendido inmediatamente y debe asistir a una clase obligatoria de educación vial. Luego de completarla, regresa a conducir con historial limpio.
    \item Si acumula \textbf{tres multas amarillas}, también es suspendido y debe completar la clase de educación vial antes de volver a conducir.
\end{itemize}

El comportamiento de José cambia con el tiempo:

\begin{itemize}
    \item Justo después de volver a conducir (es decir, con historial limpio), la probabilidad de recibir una multa es del 60\%, dividida en un 40\% para una multa amarilla y un 20\% para una roja. La probabilidad de no recibir multa es del 40\%.
    \item Por cada multa amarilla acumulada, la probabilidad total de ser multado disminuye en un 10\%, manteniéndose constante la proporción entre multas amarillas (2/3) y rojas (1/3).
\end{itemize}

\textbf{Se propone modelar este sistema como una cadena de Markov}, considerando los siguientes estados:  \textbf{M0}: José con historial limpio (0 multas), \textbf{A1}: José con 1 multa amarilla, \textbf{A2}: José con 2 multas amarillas, \textbf{A3}: José con 3 multas amarillas (entra en suspensión), \textbf{R}: José recibe una multa roja (entra en suspensión) y \textbf{S}: Estado de suspensión (ya sea por 3 amarillas o por roja); José pasa automáticamente a M0 en el siguiente período.

\begin{enumerate}
    \item[(a)] Modele esta situación como una \textbf{cadena de Markov}, definiendo la matriz de transición correspondiente.
    
    \item[(b)] Calcule la \textbf{distribución de estado estable} del sistema. Interprete el resultado: ¿qué proporción del tiempo se encuentra José en cada situación?

    \item[c)] Si el costo de cada tarjeta amarilla es C\$\,500, y cada tarjeta roja es C\$\, 2000. Calcule el costo esperado en un ciclo de multas para José. 
\end{enumerate}
\end{problem}
\vspace{-.7cm}
\begin{problem}
\textbf{Problema 2 | Estados Absorbentes}\\
Datos de la progresión de estudiantes universitarios en una universidad
particular se resumen en la siguiente matriz de probabilidades de transición:
\[
\begin{array}{c|cccccc}
      & \text{Se gradúa} & \text{Abandona} & \text{Primer año} & \text{Segundo año} & \text{Tercer año} & \text{Cuarto año}\\\hline
\text{Se gradúa} & 1.00 & 0.00 & 0.00 & 0.00 & 0.00 & 0.00\\
\text{Abandona}  & 0.00 & 1.00 & 0.00 & 0.00 & 0.00 & 0.00\\
\text{Primer año}& 0.00 & 0.20 & 0.15 & 0.65 & 0.00 & 0.00\\
\text{Segundo año}& 0.00 & 0.15 & 0.00 & 0.10 & 0.75 & 0.00\\
\text{Tercer año}& 0.00 & 0.10 & 0.00 & 0.00 & 0.05 & 0.85\\
\text{Cuarto año}& 0.90 & 0.05 & 0.00 & 0.00 & 0.00 & 0.05
\end{array}
\]

\begin{enumerate}
\item En un discurso frente a 600 estudiantes de primer año, el decano les pide que vean a su alrededor y que se den cuenta de que aproximadamente 50\% de los estudiantes que allí están no llegará al día de la graduación. ¿Su análisis del proceso de Markov soporta lo expresado por el decano? Explique.
\item Actualmente, la universidad tiene 600 estudiantes de primer año, 520 de segundo, 460 de tercero y 420 de cuarto año. ¿Qué porcentaje de los 2\,000 estudiantes que asisten a la universidad finalmente se graduará?
\end{enumerate}
\end{problem}

\end{document}
